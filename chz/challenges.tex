\chapter{Challenges}
Currently, Django is one of the most featured frameworks for database-driven
web applications but still doesn't give a full control over the user-interface
where we have several more featured frameworks like React, etc. These are
a bit harder to configure properly with Django. Below is the summary of
challenges faced during the development:

\begin{itemize}
    \item Integrating backend with a React frontend:
    
    Django templates were very less customizable and hence for the modern
    web application we need something more customizable like React. React is
    a JavaScript framework used to create some very powerful frontend
    applications. Both of Turtlemint Assistant and MintAcademy were built on
    React for the frontend and hence integrating the them was a Challenges
    due to Django being highly customized for its templates.

    \item Solving CORS issues in session based auth:
    
    Django by default uses session based authentication for the users. This
    works well for applications where both of the frontend and backend services
    are running on the same system. But having a separate frontend will cause
    CORS issue when setting up the cookie for the browser since the information
    isn't transferred between cross origin domains. To solve this we had to
    move towards JWT (Java Web Token) authentication flow.

    \item Multiple database issue:
    
    So Turtlemint have multiple databases deployed on different servers. This
    was an issue since we had to manually switch between databases to use one
    of the other databases. Django lets us specify which database to use in
    its settings.py file but if we ever had to connect to other database then
    we'll have to switch to them manually. This was causing gaps between the
    database flow, also switching was costing us some time. We had to finally
    migrate the other database into current one to solve this issue.
\end{itemize}
