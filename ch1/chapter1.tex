\chapter{Introduction}

In any service dealing with the database it becomes extremely important to
have a constant database structure in place before moving on towards the
business logic. In Django we define the service in terms of \texttt{app},
\texttt{models}, \texttt{views} and \texttt{services}. These four parts
represent the core logic service.
\texttt{Views} take care of the exchange of the request and response objects from APIs.
Usually when a API is called, a request object is sent to the server containing
information about the request being made. The server then has to return the
appropriate Response object which then the browser parses and outputs for the
user. This exchange between request and response is a part of Views.

\section{Background}
Turtlemint has a separate database which records most of the things related to
insurance policy issuance. This data is very volatile and is expected to change
every month. Due to this, it becomes harder to change the database every time
there is a change in the information. To handle this issue, the purpose of the
project is to create a new service which would wrap the information change in
terms of database calls and let the user seamlessly update the information.

\section{Significance}
The new service will be able to handle all information changes related to the
database. Moreover the service would have an integrated authentication and
authorization which allows multiple users to use this service at a time.
Previously, someone from the development team had to intervene with the data
team to manually create database queries and update accordingly. This process
was not only time consuming but also was inefficient. The new service would
solve this issue and would allow the data team itself to update the database.

\section{Method used}
The service is built using Django and Django Rest Framework (DRF) which are
two Python packages built for faster development of database-driven web
applications. Django is also open-source and allows users to modify the
report, modify any bugs if they found any. This helps for long term
support applications. Django has three major parts: \texttt{models},
\texttt{views} and \texttt{templates}. Models are used to create database
schema, views contain the business logic and templates are used for user
interface.

\section{Limitations}
Django being open-source does help is most issues. Although, since Django was
built to reduce the development time significantly it might still not have
all features of a system with independent database architecture. Django also
introduces the concept of migrations which are a set of database schema changes
maintained as a set of files. These migrations can be difficult to manage if
an applications is prone to lot of database changes.

\section{Project Structure}
Django has already defined its project structure. Every Django project has some
applications. Every applications represents set of logic related to one purpose
or business objective. Every project can have any number of applications inside
it. There is a common \texttt{settings.py} file which is used for managing
settings for all applications. The basic structure of the project can be
represented as below:

\dirtree{%
.1 project.
.2 settings.py. 
.2 app1.
.3 models.
.3 views. 
.3 templates. 
.2 app2.
.3 models.
.3 views. 
.3 templates. 
}
