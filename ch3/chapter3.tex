\chapter{Project Architecture}

\section{An overview of the project architecture}
The project objective is to ease the process of updating and managing the
database with proper authorization for the users. This adds up to few main
components of the project which are:

\begin{itemize}
    \item Database tables
    \item CRUD operations
    \item User permissions
\end{itemize}

\subsection{Database tables}
Currently in Turtlemint the entire database in nosql MongoDB. Hence, the first
challenge is to convert these nosql tables into relational postgres tables and
import all data from the previous database to current database. The number of
such tables is huge and hence some scripts are generated to copy the previous
database tables into the current ones. We will take a look at these scripts
later in the models section.

\subsection{CRUD operations}
Every table or model in a database has some basic CRUD (Create, Read, Update,
Delete) operations associated with it. These are declared in the models class
itself to avoid rewriting them again in the business logic. These operations
handle the basic interaction with the database and can be used anywhere in the
project scope.

\begin{pythoncode}
from django.db import models

def create(**kwargs):
    turtlemint_model.objects.create(**kwargs)

def read(**kwargs):
    turtlemint_model.objects.get(**kwargs)

def delete(**kwargs):
    turtlemint_model.objects.get(**kwargs).delete()

class turtlemint_model(models.Model):
    """
    Model representing turtlemint meta information.
    """
    key = models.CharField(max_length=100, unique=True, blank=True)
    name = models.CharField(max_length=100)
    description = models.TextField(max_length=100)
    ...
\end{pythoncode}

\subsection{User permissions}
There are usually two ways we can create users based on their permission
levels. One of the ways is to use Django's inbuilt \texttt{groups} which can
create separate groups for users and every group would then have a dedicated
level of permission which would then be applied for every user in that group.
This process is handy for most of the simple applications which don't have
to deal with a lot of different user sets.

The other way is to define a \texttt{CustomUser} model overriding Django's
inbuilt \texttt{User} model. This gives us much more freedom over the user
characteristics and can then be used with different permission classes. As
mentioned above, Django have some of the permission classes already defined
while other can be manually created.

\begin{pythoncode}
class TurtlemintUser(AbstractBaseUser):
    """
    This class represents Custom User Model,
    that overrides django builtin User model
    """
    key = models.CharField(max_length=100, unique=True, blank=True)
    email = models.EmailField(
        verbose_name='email address',
        max_length=255,
        unique=True,
    )
    first_name = models.CharField(max_length=100, blank=True)
    last_name = models.CharField(max_length=100, blank=True)
    password = models.CharField(max_length=200, blank=True)
    ...
\end{pythoncode}

Every user has some sort of permissions associated with it. These permissions
are used to determine whether a user can interact with a certain database table
or not. Django in itself provides \texttt{PermissionClasses} which can be used
to determine the scope for the user. We can also create custom permissions
classes by extending the \texttt{BasePermission} class and overriding
appropriate class methods.

An example of a \texttt{CustomPermission} class can be given as below:
\begin{pythoncode}
from rest_framework.permissions import BasePermission
class IsSuperAdmin(BasePermission):
    """
    Allows access only to super admin users.
    """
    def has_permission(self, request, view):
        return bool(request.user and request.user.is_staff)
\end{pythoncode}

\section{Business Logic}
Most of the architecture is covered in the above sections, there are still
some organization specific requirements which are to be fulfilled. These are
all considered under business logic. The business logic just represents the
actual code written to complete an objective or requirement imposed by the
organization. At Turtlemint, we've number of requirements for which we need
to split the project into several modules.

The most of the business logic should go in the \texttt{services/} directory.
This is not a Django specific module, but is actually user defined. Hence, the
name can be changed to whichever suitable. The business logic should be as
isolated as possible since then it will be easier to write unittest against
it. Unittest are written to check for every possible scenario and to check if
its response is the one which is expected. Having a good code coverage will
help maintainers in maintaining the software in long run.
