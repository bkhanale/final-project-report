\chapter{Preliminary}
There are three important parts of this project:
\begin{itemize}
    \item Django (for API development)
    \item Postgresql (for database requirements)
    \item Social Auth (for authentication)
\end{itemize}

\section{Django}
Django takes care of integrating the database, authentication and authorization
of the application. For this, Django asks us to construct each of models, views
and templates for the application. Its important to note that all of the three
parts are to be defined for every application in the project.

\subsection{Models}
Every model represents a table in the database. We will be forming up the
basic model structure and follow them throughout the entire project.
Every model can be defined in the \texttt{models.py} file which is recognized
by django or in the \texttt{models} module. The overall model structure would
look like this:

\dirtree{%
.1 project.
.2 app1.
.3 models.
.4 modela.py. 
.4 modelb.py. 
.4 modelc.py. 
.2 app2.
.3 models.
.4 modela.py. 
.4 modelb.py. 
}

\subsection{Views}
Every view represents the exchange of data between every API call. Every view
gets a request object and is expected to return a response object. The request
contains all information regarding the API call including the user who
initiated it. Also all of the required parameters for the API call are passed
in the same request object.\\
Views too can be distributed across multiple files for convenience.

\subsection{Templates}
Templates are used for the user interface which are automatically rendered by
Django in html. By default, Django comes with the \texttt{admin} template.
The admin page on Django lets user to modify any of the existing models and
settings. Multiple templates can be added representing the user interface.
The templates are also convenient in authentication since we can use variables
inside templates.

\section{Postgresql}
We've used Postgresql for out database requirements. Because postgres can
scale on high requirement environment and is one of the best options for
production databases. The postgresql can be installed as a separate package
and is available on all platforms including Linux, Windows and MacOS making
it ideal for deployment and development purpose.

Postgres related settings can be configured inside \texttt{settings.py} file
inside project directory. We can also specify the requirements as a part of the
environment file to avoid exposing the credentials. The database setting is
generic in Django meaning you can use the database identically as any other
database driver. Hence, it is very easy to switch between database if we ever
wanted to.

\section{Social Auth}
For authentication there is a separate package known as
\textit{Python Social Auth} which takes care of our social authentication.
Social authentication is important because many users now try to avoid
remembering username and password and hence, using one of the social auth
method will bypass the need to remember usernames and passwords. This
authentication is based on one of the social media platforms including Google,
Facebook, Twitter, etc. These platforms proving authentication flow known as
\textit{OAuth}. The common user fields like email address, name are
automatically obtained from these platforms and a \textit{token} is created
for the user to communicate with the application.

Everyone in Turtlemint have their email with domain \textit{turtlemint.com}
which lets us use Google OAuth2 to authenticate the user before logging it
on our application. The email address and name is obtained from Google and
appropriate permissions are issued to the user based on his access level.
